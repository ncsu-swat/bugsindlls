\documentclass[sigconf]{acmart}

\usepackage{multirow}
\usepackage{xcolor}
%%
%% \BibTeX command to typeset BibTeX logo in the docs
\AtBeginDocument{%
  \providecommand\BibTeX{{%
    Bib\TeX}}}

%% Rights management information.  This information is sent to you
%% when you complete the rights form.  These commands have SAMPLE
%% values in them; it is your responsibility as an author to replace
%% the commands and values with those provided to you when you
%% complete the rights form.
\setcopyright{acmlicensed}
\copyrightyear{2018}
\acmYear{2018}
\acmDOI{XXXXXXX.XXXXXXX}
%% These commands are for a PROCEEDINGS abstract or paper.
\acmConference[Conference acronym 'XX]{Make sure to enter the correct
  conference title from your rights confirmation email}{June 03--05,
  2018}{Woodstock, NY}
%%
%%  Uncomment \acmBooktitle if the title of the proceedings is different
%%  from ``Proceedings of ...''!
%%
%%\acmBooktitle{Woodstock '18: ACM Symposium on Neural Gaze Detection,
%%  June 03--05, 2018, Woodstock, NY}
\acmISBN{978-1-4503-XXXX-X/2018/06}


%%
%% Submission ID.
%% Use this when submitting an article to a sponsored event. You'll
%% receive a unique submission ID from the organizers
%% of the event, and this ID should be used as the parameter to this command.
%%\acmSubmissionID{123-A56-BU3}

%%
%% For managing citations, it is recommended to use bibliography
%% files in BibTeX format.
%%
%% You can then either use BibTeX with the ACM-Reference-Format style,
%% or BibLaTeX with the acmnumeric or acmauthoryear sytles, that include
%% support for advanced citation of software artefact from the
%% biblatex-software package, also separately available on CTAN.
%%
%% Look at the sample-*-biblatex.tex files for templates showcasing
%% the biblatex styles.
%%

%%
%% The majority of ACM publications use numbered citations and
%% references.  The command \citestyle{authoryear} switches to the
%% "author year" style.
%%
%% If you are preparing content for an event
%% sponsored by ACM SIGGRAPH, you must use the "author year" style of
%% citations and references.
%% Uncommenting
%% the next command will enable that style.
%%\citestyle{acmauthoryear}

%% macros
\newcommand{\Fix}[1]{\textbf{[[}\textcolor{red}{#1}\textbf{]]}}
\newcommand{\tname}{BugsInDLLs}
\newcommand{\jax}{JAX}
\newcommand{\torch}{PyTorch}
\newcommand{\tf}{Tensorflow}
\newcommand{\numbugs}{115}

%%
%% end of the preamble, start of the body of the document source.
\begin{document}

%%
%% The "title" command has an optional parameter,
%% allowing the author to define a "short title" to be used in page headers.
\title{\tname: A Database of Reproducible Bugs in Deep
  Learning Libraries to Enable Systematic Evaluation of Testing
  Techniques}

%%
%% The "author" command and its associated commands are used to define
%% the authors and their affiliations.
%% Of note is the shared affiliation of the first two authors, and the
%% "authornote" and "authornotemark" commands
%% used to denote shared contribution to the research.
%% \author{Ben Trovato}
%% \authornote{Both authors contributed equally to this research.}
%% \email{trovato@corporation.com}
%% \orcid{1234-5678-9012}
%% \author{G.K.M. Tobin}
%% \authornotemark[1]
%% \email{webmaster@marysville-ohio.com}
%% \affiliation{%
%%   \institution{Institute for Clarity in Documentation}
%%   \city{Dublin}
%%   \state{Ohio}
%%   \country{USA}
%% }

%% \author{Lars Th{\o}rv{\"a}ld}
%% \affiliation{%
%%   \institution{The Th{\o}rv{\"a}ld Group}
%%   \city{Hekla}
%%   \country{Iceland}}
%% \email{larst@affiliation.org}

\author{M. M. Abid Naziri}
\affiliation{%
  \institution{NC State University}
  %\city{Rocquencourt}
  \country{USA}
}
\email{mnaziri@ncsu.edu}

\author{Aman Kumar Singh}
\affiliation{%
 \institution{Amrita Vishwa Vidyapeetham}
 %% \city{Doimukh}
 %% \state{Arunachal Pradesh}
 \country{India}}
\email{amanks@am.amrita.edu}

\author{Feiran Qin}
\affiliation{%
 \institution{NC State University}
 %% \city{Doimukh}
 %% \state{Arunachal Pradesh}
 \country{USA}}
\email{fqin2@ncsu.edu}

\author{Benjamin Wu}
\affiliation{%
 \institution{Purdue University}
 %% \city{Doimukh}
 %% \state{Arunachal Pradesh}
 \country{USA}}
\email{wu2059@purdue.edu}

\author{Saikat Dutta}
\affiliation{%
 \institution{Cornell University}
 %% \city{Doimukh}
 %% \state{Arunachal Pradesh}
 \country{USA}}
\email{saikatd@cornell.edu}

\author{Marcelo d'Amorim}
\affiliation{%
 \institution{NC State University}
 %% \city{Doimukh}
 %% \state{Arunachal Pradesh}
 \country{USA}}
\email{mdamori@ncsu.edu}


%% \author{Julius P. Kumquat}
%% \affiliation{%
%%   \institution{The Kumquat Consortium}
%%   \city{New York}
%%   \country{USA}}
%% \email{jpkumquat@consortium.net}

%%
%% By default, the full list of authors will be used in the page
%% headers. Often, this list is too long, and will overlap
%% other information printed in the page headers. This command allows
%% the author to define a more concise list
%% of authors' names for this purpose.
\renewcommand{\shortauthors}{Trovato et al.}


%%
%% The abstract is a short summary of the work to be presented in the
%% article.
\begin{abstract}
  \sloppy
  AI-enabled applications are prolific today. Deep Learning
  libraries~(DLLs), such as \torch{} and \tf{}, provide the building
  blocks for the AI components of these applications. As any piece of
  software, these libraries can be buggy.
  %% and those bugs can affect a
  %% great deal of applications using those libraries.
  An impressive number of bug-finding techniques to address this
  problem, but the lack of a curated set of reproducible bugs in DLLs
  hinders credible evaluation of such techniques. We present \tname, a
  database of curated reproducible bugs to fill that gap.
  %% enable credible  evaluation of bug-finding DLL testing techniques.
  Our dataset currently consists of \numbugs{} environments to
  reproduce bugs across three popular DLL libraries, namely, \jax,
  \tf, and \torch.
\end{abstract}

%%
%% The code below is generated by the tool at http://dl.acm.org/ccs.cfm.
%% Please copy and paste the code instead of the example below.
%%
%% \begin{CCSXML}
%% <ccs2012>
%%  <concept>
%%   <concept_id>00000000.0000000.0000000</concept_id>
%%   <concept_desc>Do Not Use This Code, Generate the Correct Terms for Your Paper</concept_desc>
%%   <concept_significance>500</concept_significance>
%%  </concept>
%%  <concept>
%%   <concept_id>00000000.00000000.00000000</concept_id>
%%   <concept_desc>Do Not Use This Code, Generate the Correct Terms for Your Paper</concept_desc>
%%   <concept_significance>300</concept_significance>
%%  </concept>
%%  <concept>
%%   <concept_id>00000000.00000000.00000000</concept_id>
%%   <concept_desc>Do Not Use This Code, Generate the Correct Terms for Your Paper</concept_desc>
%%   <concept_significance>100</concept_significance>
%%  </concept>
%%  <concept>
%%   <concept_id>00000000.00000000.00000000</concept_id>
%%   <concept_desc>Do Not Use This Code, Generate the Correct Terms for Your Paper</concept_desc>
%%   <concept_significance>100</concept_significance>
%%  </concept>
%% </ccs2012>
%% \end{CCSXML}

%% \ccsdesc[500]{Do Not Use This Code~Generate the Correct Terms for Your Paper}
%% \ccsdesc[300]{Do Not Use This Code~Generate the Correct Terms for Your Paper}
%% \ccsdesc{Do Not Use This Code~Generate the Correct Terms for Your Paper}
%% \ccsdesc[100]{Do Not Use This Code~Generate the Correct Terms for Your Paper}

%%
%% Keywords. The author(s) should pick words that accurately describe
%% the work being presented. Separate the keywords with commas.
\keywords{Deep learning libraries, testing, benchmarking}
%% %% A "teaser" image appears between the author and affiliation
%% %% information and the body of the document, and typically spans the
%% %% page.
%% \begin{teaserfigure}
%%   \includegraphics[width=\textwidth]{sampleteaser}
%%   \caption{Seattle Mariners at Spring Training, 2010.}
%%   \Description{Enjoying the baseball game from the third-base
%%   seats. Ichiro Suzuki preparing to bat.}
%%   \label{fig:teaser}
%% \end{teaserfigure}

%% \received{20 February 2007}
%% \received[revised]{12 March 2009}
%% \received[accepted]{5 June 2009}

%%
%% This command processes the author and affiliation and title
%% information and builds the first part of the formatted document.
\maketitle

\section{Introduction}

\sloppy Serveral important application domains (e.g., transportation,
medical diagnosis, software development) use AI to optimize some
element of their business process. Deep Learning libraries~(DLLs)
(e.g., \jax, \torch{}, and \tf{}) provide the building blocks for the
AI components of these applications.  Unfortunately, as any piece of
software, these libraries contain bugs. An impressive number of
techniques have been recently proposed to find bugs in these
libraries~\Fix{cite 10+ fuzzing papers here}, however we observe that
the evaluation of these techniques do \emph{not} use a reference set
of reproducible bugs to evaluate their effectiveness. The lack of
evaluation standards is an obstacle to evaluate progress in this
important field.

This paper presents \tname, a database of curated reproducible bugs to
enable credible evaluation of DLL testing techniques.  Our dataset
currently consists of \numbugs{} environments to reproduce bugs across
three popular DLL libraries, namely, \jax, \tf, and \torch.

\Fix{...}


\begin{enumerate}

\sloppy  
\item \textbf{Handling bugs from nighly builds require saving wheels.}
  Many of the reproducible bugs rely on nightly builds since they can
  not be reproduced with any release builds. These nightly builds have
  a limited amount of time when they are publicly available, and they
  are de-indexed and removed after that time. In order to ensure bugs
  that depend on these nightly builds can be reproduced after these
  builds disappear, we save the whl files of these builds.

\item \textbf{Handling bugs in specific GPU-CUDA versions require OS
  changes:} Many of the GPU-related bugs can only be repro
  duced with
  specific GPU builds and specific CUDA builds. Changing different
  virtual environments does not change the CUDA versions of the
  system. That is why we use Docker when a bug relies on a specific
  version of CUDA to be installed in the system.

\end{enumerate}


Table~\ref{...}

\begin{table}
  \centering
  \caption{Characterization of reproducible bugs from \tname.}
\begin{tabular}{lrrr}
  \toprule
  & \multirow{2}{*}{\#} &  \multicolumn{1}{c}{build}          &   \multicolumn{1}{c}{enviroment}  \\
  &   & release-nightly &  venv-docker  \\
  \cmidrule(lr){2-4}
  \jax{} & 44 & 43-1 &  43-1 \\
  \torch{} & 33 & 15-18 & 19-14 \\
  \tf{} & 25 & 24-1 & 25-0 \\
  \midrule
  \multicolumn{1}{c}{$\Sigma$} & 102 & 82-20 & 87-15 \\
  \bottomrule
\end{tabular}
\end{table}


\section{Methodology}
\subsection{DLL and Bug Selection}
\subsection{Reproducing Bugs}
\subsection{}

\section{\tname: Database of Bugs in DLLs}
\subsection{Bug Interface}
\subsection{Test Oracle}
\subsection{Usage and Setup}

\section{\tname: Use cases}

\subsection{Evaluating DLL testing techniques}
\subsection{Fault Detection}
\subsection{Program Repair}


\section{Related Work} 

\section{Conclusion and Future Work}

\bibliographystyle{ACM-Reference-Format}
\bibliography{references}

%%
%% If your work has an appendix, this is the place to put it.
\appendix

\section{Research Methods}


\end{document}
\endinput
%%
%% End of file `sample-sigconf.tex'.
